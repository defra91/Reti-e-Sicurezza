\section{Capitolo 3}

\subsubsection{Che cos'è il byte stuffing?}

Il byte stuffing è usato in PPP (Point-to-Point Protocol) ed è un metodo usato per capire dove inizia e finisce un frame. Il byte stuffing inserisce prima e dopo ogni frame un byte, chiamato flag byte.
Quindi in caso di perdita di dati, basterà cercare l'ultimo flag byte caricato. Un possibile
inconveniente è che dentro i dati ci sia un flag byte. In questo caso, basta che la sorgente inserisca un byte di Escape subito prima di ogni occorrenza e la destinazione provvederà a toglierli.
Il controllo del frame è eseguito dallo strato Data Link, il quale prima di passa passare il frame allo strato successivo, elimina le sequenze escape e i flag byte.

Se trova una sequenza di escape nel frame, devo inserire a sua volte un altro escape per non far
ignorare le successiva. Un difetto di questo metodo è legato all’uso di caratteri a 8 bit. Ad esempio Unicode usa una codifica a 16bit.

\subsubsection{Descrivere il Bit stuffing}

È analogo al byte stuffing, solo che è fatto a livello di bit, così viene aggirato il problema del byte stuffing e quindi si può scegliere la dimensione della flag. Ogni frame inizia e finisce con 01111110 (protocollo X.25). Ogni volta che lo strato datalink della sorgente incontra 5, uni di fila inseriscono uno zero. Il destinatario ogni volta che incontra 5 uni, elimina lo zero successivo. Questo metodo è usato anche per fare padding.

\subsubsection{Numero di bit necessari per riconoscimento(correzione) degli errori di trasmissione}

Esistono due strategie per la gestione degli errori:

\begin{itemize}

\item Codifica a correzione d’errore: vengono inclusi nel blocco trasmesso una quantità di
informazioni ridondanti, che in caso di errore permettono di ricostruire e riparare il frame;
\item Codifica a rilevazione di errori: introduce ridondanze in modo che il destinatario capisca se
c’è un errore, ma NON è in gradi di correggerlo.

\end{itemize}

Dati m bit per il frame e r bit ridondanti: $m+r=n$, ovvero n è la lunghezza totale del messaggio
inviato, chiamato codeword. Prendiamo ad esempio due codeword: 810001001 e 10110001. Come determino quanti bit corrispondenti sono differenti? Eseguo l’OR esclusivo ed ottengo 0011000. Il numero di bit corrispondenti diversi è detto distanza di Hamming: se due codeword sono a distanza di Hamming d uno dall’altra, sono necessari d errori su singoli bit per convertire una sequenza nell’altra. Per trovare d errori necessito di codifica con distanza d+1. 
Per correggere d errori necessito di codifica con distanza 2d+1.

\subsubsection{Si descriva cos'è il CRC (Cycle redundancy check). Si calcoli inoltre il CRC di 10011101 usando il polinomio generatore di $x^4+x+1$.}

Il cyclic redundancy check è un metodo per il calcolo di checksum. Il nome deriva dal fatto che i dati d'uscita sono ottenuti elaborando i dati di ingresso, i quali vengono fatti scorrere ciclicamente in una rete logica. Il controllo CRC è molto diffuso perché la sua implementazione binaria è semplice da realizzare, richiede conoscenze matematiche modeste per la stima degli errori e si presta bene a rilevare errori di trasmissione su linee affette da elevato rumore di fondo. 16) il generatore deve avere i bit di ordine più alto e più basso uguali a 1. per poter calcolare il checksum di un frame di m bit che corrisponde al polinomio M(x), il frame deve essere più lungo del polinomio generatore.

Quando la destinazione riceve un frame prova a dividerlo per il polinomio generatore. Se la
divisione ha un resto, c'è stato un errore nella trasmissione.

\begin{enumerate}

\item Posto r il grado di G(x), aggiungere r bit con valore zero dopo la parte di ordine piu'
basso del frame. Così che adesso contenga m+r bit e corrisponda al polinomio $x^r M(x)$;
\item Dividere la sequenza di bit corrispondenti a G(x) per la sequenza corrispondente a $x^r
M(x)$ usando la divisione modulo 2;
\item Sottrarre il resto dalla sequenza corrispondente a $x^r M(x)$ usando la sottrazione in
modulo 2. Il risultato è il frame con checksum pronto per la trasmissione.

\end{enumerate}

Dati due polinomi P e G (generatore) dobbiamo aggiungere alla destra di P tanti zeri quanto è il
grado massimo di G e otteniamo il polinomio F. Poi si dive il polinomio ottenuto per G, si ottiene un resto che va sommato al polinomio F, infine si raggruppano i bit in gruppetti di 4 e si codifica in esadecimale.

Esempio: $P=10011101$ e $G= x^4+x+1$, quindi abbiamo $G=10011$, $F= P=100111010000$ e il resto è $1111$.

Il polinomio finale è $1001 1101 1111$ in esadecimale è $9DF$.

\subsubsection{Descrivere Il protocollo stop and wait, pregi e difetti}

Il protocollo S\&W è un protocollo molto semplice per il controllo del flusso e si può utilizzare in
canali simplex o half-duplex. Quando il mittente invia un blocco aspetta che il ricevente invii una
conferma, un ACK (acknowledge). Lo svantaggio principale è l'attesa, ma in compenso non c'è
bisogno di regolare la velocità.

Possono sorgere due errori: il frame non arriva mai a destinazione e il mittente aspetta e rinvia
all'infinito: c'è bisogno di un tempo limite di rinvio. L'altro problema riguarda l'ACK: potrebbe non arrivare al mittente, il quale rinvia il pacchetto e al destinatario arriva più volte, ma per fortuna viene scartato, grazie al numero di messaggio.

\subsubsection{Cos’è il piggybacking}

La tecnica consiste nello sfruttare un messaggio del destinatario al mittente come passaggio per
l’ACK, in modo da non perdere tempo e sfruttare al meglio il canale di comunicazione(un messaggio
in meno da inviare). Il campo ACK è posto all’inizio del frame. Il problema principale è quando fare
piggybacking: in attesa molto lunga può essere vana, poiché il mittente rinvia il frame. Quindi se il pacchetto arriva per il mittente è caricato e inviato in tempo breve si fa piggybacking, altrimenti s’invia l’ACK separatamente.

\subsubsection{Si descriva la tecnica dello Sliding window}

È un protocollo per il controllo di flusso. Utilizza la tecnica del piggybacking. Invia pacchetti e
aspetta messaggio di conferma ACK. Sorge il problema di quando fare il piggybacking, un’attesa
troppo lunga può rendere vano il tutto perché il mittente fa un rinvio del frame. Quindi se il
pacchetto arriva velocemente viene fatto piggybacking sennò viene inviato separatamente.
L’essenza del protocollo è che ogni partecipante alla comunicazione deve tener sotto controllo 2
finestre, quella dei frame in entrata e quella dei frame in uscita. Ogni frame in usciata contiene un numero di sequenza e il destinatario deve tener traccia di questi per la ricezione mentre il mittente per l’invio.

Con lo sliding window a 1 bit, viene utilizzato il metodo stop and wait. Quando il mittente invia un
frame, resta nella finestra finché non viene ricevuto l’ACK corrispondente prima di aggiornare la
finestra. I frame inviati sono numerati con 1 o 0. Quando il destinatario riceve il frame, controlla
che il numero sia uguale a quello che aspettava, se s’invia l’ACK. Se l’ACK contiene il numero che la sorgente si aspettava allora continua a inviare, altrimenti re invia quello segnato nel buffer. Si può utilizzare anche il pipelining, inviando più frame contemporaneamente prima di entrare in attesa. Il destinatario aggiorna la finestra non appena riceve il frame e invia l’ACK. Esistono 2 approccio: go back n e selective repeat.

\subsubsection{Si descriva l'idea dei protocolli ``go back N'', indicandone pregi e difetti.}

Questo protocollo è utilizzato con sliding window di ampiezza 1 in ricezione e maggiore di uno in
invio. I pacchetti arrivano uno per volta e su di essi viene fatto un checksum, se si trovano errori
vengono segnalati alla sorgente indicando il numero del pacchetto danneggiato. Per questo motivo la finestra deve essere capiente. Se la finestra sorgente si riempie prima che il timer di arrivo scatti, la pipeline viene svuotata. La destinazione intanto scarta i pacchetti successivi a quello in errore.

Questo approccio è efficacie contro la prevenzione di errori, ma occupa molta banda se la frequenza di errori è alta.

\subsubsection{Si descriva cos'è la tecnica del selective repeat}

La tecnica del selective repeat è una tecnica che si usa con il protocollo sliding window. In questo caso il buffer della destinazione deve essere più capiente. Infatti in caso di errori, viene inviato alla sorgente un NACK, indicandone il pacchetto. Finché il pacchetto contenente l'errore non arriva al destinatario, i pacchetti successivi vengono mantenuti nel buffer. Una volta arrivato tutto, il messaggio viene passato allo strato network. Inoltre la sorgente dispone di un timer per cui, se il pacchetto è errato e non arrivasse un NACK il pacchetto sarebbe rinviato comunque.
