\section{Capitolo Strato Fisico}

\subsection{Cosa si intende per serie di Fourier}

Le informazioni possono essere trasmesse via cavo variando alcune proprietà fisiche, come la
tensione e corrente. Fourier condusse alcuni studi ed arrivò alla conclusione che le informazioni
trasmesse via cavo potevano essere rappresentate da una funzione $f(t)$. Questa funzione è
composta da una serie infinita di somme di seni e coseni, ed è in grado di rappresentare un segnale
periodico e regolare. La trasmissione però non è mai perfetta e c'è per forza attenuazione di linea.
L'intervallo di frequenze trasmesse senza forte attenuazione è detto \textbf{banda passante}.
Anche in un ipotetico canale perfetto, ovvero senza attenuazioni, la velocità di trasmissione non
può essere troppo elevata; la massima velocità è data dall'equazione di Nyquist/Shannon:

\begin{center}

$V_{max} = \log_{2} V  bit/sec$

\end{center}

\subsection{Bitrate e baudrate}

Bitrate: Velocità di trasmissione si indica in $bit/s$. Il teorema di Nyquist mette in relazione il bitrate con la banda disponibile:

\begin{center}

	$2H\log_2V$

\end{center}

Con H banda disponibile e V livelli di segnale (simboli) usati:

\begin{center}

	$S/N = segnale/rumore$,  $SNR = 10\log_{10} (S/N) dB$
	\newline
	$Massimo bitrate = 2\log_2 (1+(S/N))$

\end{center}	

Baudrate: numero di imboli al secondo, un simbolo può valere più bit.

