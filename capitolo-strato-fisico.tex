\section{Capitolo Strato Fisico}

\subsection{Cosa si intende per serie di Fourier}

Le informazioni possono essere trasmesse via cavo variando alcune proprietà fisiche, come la
tensione e corrente. Fourier condusse alcuni studi ed arrivò alla conclusione che le informazioni
trasmesse via cavo potevano essere rappresentate da una funzione $f(t)$. Questa funzione è
composta da una serie infinita di somme di seni e coseni, ed è in grado di rappresentare un segnale
periodico e regolare. La trasmissione però non è mai perfetta e c'è per forza attenuazione di linea.
L'intervallo di frequenze trasmesse senza forte attenuazione è detto \textbf{banda passante}.
Anche in un ipotetico canale perfetto, ovvero senza attenuazioni, la velocità di trasmissione non
può essere troppo elevata; la massima velocità è data dall'equazione di Nyquist/Shannon:

\begin{center}

$V_{max} = \log_{2} V  bit/sec$

\end{center}

\subsection{Bitrate e baudrate}

Bitrate: Velocità di trasmissione si indica in $bit/s$. Il teorema di Nyquist mette in relazione il bitrate con la banda disponibile:

\begin{center}

	$2H\log_2V$

\end{center}

Con H banda disponibile e V livelli di segnale (simboli) usati:

\begin{center}

	$S/N = segnale/rumore$,  $SNR = 10\log_{10} (S/N) dB$
	\newline
	$Massimo bitrate = 2\log_2 (1+(S/N))$

\end{center}	

Baudrate: numero di imboli al secondo, un simbolo può valere più bit.


\subsection{Descrivere i vari tipi di cavo e confrontarli}

Principalmente esistono 3 tipi di cavo, il classico \textbf{doppino}, il \textbf{cavo coassiale} e la \textbf{fibra}.

Il doppino è composto da due conduttori di rame isolati, attorcigliati tra loro in modo
elicoidale (DNA style), per evitare interferenze fra di loro. Il doppino è molto utile per la linea
telefonica, dato che può percorrere molti km senza che il segnale si indebolisca, ovvero senza
bisogno di un'amplificazione.

Il cavo coassiale è più grosso e può estendersi per distanze maggiori rispetto il doppino. La distanza
maggiore è frutto di una maggior schermatura a cui è sottoposto il nucleo in rame del cavo che lo
rende immune dal rumore. Esistono due cavi coassiali, uno da 50 Ohm per le trasmissioni digitali e
uno da 75 Ohm per quelle analogiche.

La fibra ottica è formata da 3 parti: sorgente luminosa, mezzo di trasmissione e rilevatore di luce. La
sorgente di luce è rappresentata da LED oppure laser, anche se il secondo, oltre ad essere meno
diffuso è anche più costoso. Il mezzo trasmissivo è la fibra, composta un nucleo di vetro di pochi
micron, avvolta in una guaina di vetro rivestita a sua volta da una guaina di plastica. La luce che
attraversa la fibra è riflessa al suo interno, da un'estremità all'altra del cavo. Nonostante si
trasmetta alla velocità della luce, quest’ultima viene stroncata dalla velocità di decodifica inferiore
che avviene alle estremità. La fibra può contenere più raggi che si differenziano per l'angolo di riflessione. Questo tipo di fibra è detto multimodale. Se la trasmissione all'interno della fibra è
unica, si ha una trasmissione in linea retta, detta monomodale.

Lo svantaggio della fibra rispetto al doppino e cavo coassiale è il costo maggiore e la difficoltà
nell'unire vari pezzi di cavo, mentre per gli altri due tipi basta attorcigliare il nucleo di rame. Il
vantaggio della fibra è la manutenzione, essendo vetro è pari a zero. Altro vantaggio è l'unione di
più canali, che avviene tramite prismi.

\subsection{Caratteristiche e confronto fra i vari tipi di satellite, GEO, MEO, LEO}

Un satellite è un grande ripetitore di microonde posizionato in cielo. Ci sono tre tipi di satelliti che si
differenziano per la loro distanza dalla superficie terrestre.

I satelliti più lontani sono detti \textbf{geostazionari} e sono posti in successione su un’orbita circolare al livello dell’equatore, ad una distanza minima di 2 gradi uno dall'altro( ps: immaginare tanti cerchi
concentrici che hanno come primo cerchio il nostro equatore e tutti gli altri più grandi, i satelliti
sono su uno di questi). Questi satelliti sono molto lontani dalla terra e per questo hanno un tempo
medio di ritardo della trasmissione di 300 millisecondi, ma con uno di essi possiamo coprire quasi
un terzo della superficie terrestre. I satelliti \textbf{LEO} distano circa 500 km dalla terra, hanno un tempo di latenza inferiore rispetto ai GEO, come il consumo energetico. Essendo vicini, per coprire tutta la
terra, sono necessari molti satelliti. Si muovono velocemente. I satelliti \textbf{MEO} sono posti a un'orbita intermedia tra i LEO e GEO, hanno una velocità relativamente bassa, in quanto sono posti a 18 mila
km dalla terra e il loro tempo di rivoluzione è di 6 ore.

\subsection{Che cos'è la modulazione in frequenza (FM)? E in ampiezza(AM)?}

La modulazione in frequenza è una delle tecniche di trasmissione utilizzate per inviare informazioni
attraverso la variazione della frequenza dell’onda portante. La FM è una modulazione a onda
continua, ovvero viene modulata la portante sinusoidale.
Per riuscire a inviare dati in forma digitale è necessario un ampio spettro di frequenza, questo
rende adatta la trasmissione in banda base solo a basse velocità e a distanze brevi. Nella
modulazione a frequenza vengono utilizzate 2 o più frequenze.

\textbf{Pro}:

\begin{itemize}

\item Molto meno sensibile ai disturbi rispetto all’AM;
\item Permette una trasmissione di miglior qualità;
\item Efficienza energetica molto maggiore, cioè il segnale di informazione non richiede potenza aggiuntiva per essere trasmesso.
\item 

\end{itemize}

\textbf{Contro}:

\begin{itemize}

\item Necessità di circuiti molto più complessi;
\item Occupa più banda;
\item 

\end{itemize}

Modulazione in ampiezza (AM): due diverse ampiezze sono usate per rappresentare 0 e 1. Utilizza
per il segnale un segnale a radio frequenza come portante. L’AM modifica il segnale in modo
proporzionale.

\textbf{Pro}:

\begin{itemize}

\item Semplice da mettere in pratica.

\end{itemize}

\textbf{Contro}:

\begin{itemize}

\item Molto sensibile a disturbi.

\end{itemize}

\subsection{Che cos'è la modulazione delta (delta modulation)?}

Questa tecnica è una differente tecnica di multiplexing (più conversazioni nello stesso mezzo fisico)
a divisione di tempo. Ogni valore campionato differisce dal precedente di +1 o -1 sotto le condizioni
che può essere trasmesso un singolo bit che dice se il nuovo campione è maggiore o minore del
precedente.

\subsection{Descrivere in dettaglio il GSM (Global System for Mobile connection)}

Il GSM è una tecnologia simile al D-AMPS, appartenente alla seconda generazione di cellullari con
qualche modifica. La prima sostanziale è il numero di canali, infatti il GSM ha 124 coppie di canali
simplex ampi 200KH e supporta fino ad 8 connessioni contemporanee grazie al multiplexing a
divisione di tempo. La trasmissione e la ricezione non avvengono nello stesso intervallo, poiché il
sistema non è in grado di gestirlo. Il GSM è il protocollo che ha introdotto le SIM, le quali
contengono IMSI e la chiave crittografia KI, diversa per ogni SIM. Il cellulare manda IMSI e KI in
broadcast. L'operatore riceve entrambi e invia un numero casuale, che viene analizzato e
rimandato con la firma del KI all'operatore.

La struttura a cella GSM: nel protocollo GSM ci sono 4 tipi di celle: macro, micro, pico e Umbrella.
Le prime sono le più grandi, sono sopraelevate rispetto gli edifici e hanno un raggio massimo di 35
km. Le micro sono più piccole, coprono un'altezza pari agli edifici. Le pico sono molto piccole, usate
in aree molto dense, tipicamente indoor. Umbrella è una piccola estensione, usata per coprire i
buchi tra le varie celle sopracitate.

\subsection{Si descriva la tecnica CDMA (Code Division Multiple Access), possibilmente con esempio}

CDMA permette la trasmissione per tutto il tempo attraverso l'intero spettro. Queste trasmissioni
multiple e simultanee vengono separate tramite tecnica di codifica. L'idea è che i segnali si
sommino linearmente, ma ogni coppia lo fa in lingua diversa. Per risalire a ciò che viene detto,
basta togliere il rumore aggiunto dalle altre conversazioni utilizzando le matrici di Walsh. Vediamo
un esempio.

Creando una matrice di Hadamard 4x4 posso gestire 4 lingue, invertendo ogni riga ottengo altre 4
parole, in modo da avere una coppia di parole per ogni lingua. Ognuno usa una parola, si sommano
le coordinate di ogni parola ottenendo un unico vettore risultante che moltiplicato per una parola
di una determinata lingua, fornisce un numero:

\begin{itemize}

\item \textbf{Zero}: se il dispositivo non ha trasmesso;
\item \textbf{Positivo}: c'è una parola in quella lingua e la parola è la parola positiva;
\item \textbf{Negativo}: c'è una parola in quella lingua e la parola è la parola negativa;

\end{itemize}

Es: Si costruisca una base trasmissiva (chip codes) per 18 stazioni in CDMA (volendo, usando le
matrici di Hadamard).
Basta fare la matrice di hadamard 32x32 e prendere solo 18 righe Il chip codes è una riga della
matrice (di Hadamard) che viene assegnata alla singola stazione che trasmette quello per mandare
un 1 o il complemento a 1 $(riga * -1)$ per mandare uno 0. Ogni riga definisce una ``lingua'' diversa che
è linearmente indipendente dalle altre (alias riga $S*T = S*(-1*T )= 0$ se $S != T$).

\subsection{Il GPRS: cos'è, difetti e pregi}

Il GPRS è un'evoluzione del GMS che permette la gestione del traffico a pacchetti. Al contrario del
GSM non serve un servizio dedicato ma vi è un canale condiviso. Lo spreco di banda è inesistente e
si utilizza una tariffa a traffico e non a tempo, come avviene per il GSM. IL GPRS aggiunge il
supporto a PPP e IP. essendo una naturale evoluzione del GSM, ci furono differenti classi di
cellulare, a seconda del supporto alla prima o seconda tecnologia.

Nei cellulari in classe C, l'utente deve selezionare quale comunicazione utilizzare, se GSM oppure
GPRS. La classe B, permette di utilizzare entrambe le reti, ma se si sta scaricando un pacchetto e si
riceve una chiamata, il download viene sospeso. Prima della classe A, esiste una pseudo classe A, in
cui si possono usare contemporaneamente utilizzando una solo frequenza. La Classe A, permette di
utilizzare sia una che l'altra tecnologia, contemporaneamente, è come avere due cellulari
indipendenti.

La sicurezza è analoga al GSM, con l’aggiunta di una seconda chiave Kc( cipher key). Questa è
generata ogni volta dalla Ki e da un numero casuale ogni volta che l’utente di autentica.

\subsection{Handoff: che cos’è e i vari tipi}

Nelle connessioni mobili, ogni telefono è connesso alla rete tramite una sola cella finché non si
sposta. Quando ci si sposta, si deve cambiare la cella precedente con una più vicina, anche per
evitare problemi dati dalla distanza. La disconnessione da una cella, può avvenire con due modalità:

\begin{itemize}

\item Hard handoff: quando il segnale è troppo debole, lo switching office chiede alle celle vicine
quanta potenza ricevono dal cellulare. Queste gli rispondono e il cellulare viene riassegnato
alla cella con più potenza. Quindi il cellulare viene mollato e poi riagganciato, in qualche
caso è presente del lag che fa cadere la linea;
\item Soft handoff: introdotto da GSM per ovviare al problema del lag, quando il cellulare ha
poco segnale dalla cella, prima di lasciarla, si aggancia ad una nuova e poi abbandona la
vecchia. Occorre che il cellulare gestisca due frequenze, cosa che 1G e 2G non supportavano.

\end{itemize}

\subsection{FDM, TDM, CDM: algoritmi per la selezione della banda}

FDM: sfrutta la trasmissione in banda passante per condividere un canale, divide lo spettro in
bande di frequenza di cui ogni utente ha uso esclusivo per inviare il proprio segnale.

TDM: gli utenti fanno a turni secondo una politica round-robin e ognuno di loro, periodicamente
prende possesso della banda completa per un tempo limitato.

CDM: comunicazione a spettro distribuito in cui un segnale a banda stretta viene sparso su una
banda di frequenza più ampia. Ciò rende il segnale più tollerante alle interferenze e permette a più
segnali di utenti diversi di condividere la stessa banda di frequenza, chiamato anche CDMA.

\subsection{QAM e QAM16 (Quadrature Amplitude Modulation)}

È un sistema di modulazione numerica sia analogica che digitale. Le portanti sono solitamente
sinusoidali. Il termine quadratura indica che gli angoli differiscono di $90 ^{\circ}$. Il segnale può essere visto
come la somma di due segnali modulati in fame.

QAM: Più immune al rumore si ottiene tramite i diagrammi a costellazione, quelli circolari sono
quelli ideali ma sono più difficili sia da ottenere che da decodificare.
QAM 16: Quando si voleva spingere sull’acceleratore, nella trasmissione di dati via cavo, si è
pensato che il miglior approccio da utilizzare era combinare due tipi di modulazione assieme,
l’ampiezza e la fase.

Da questa idea nasce il QAM-16. Grazie ad esso possiamo utilizzare un alfabeto più ampio e spedire
un simbolo su 16 ogni unità di tempo con bitrate quadruplo.