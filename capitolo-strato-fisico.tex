\section{Lo strato fisico}

\subsection{Mezzi di Trasmissione}

\subsubsection{Basi teoriche della comunicazione dati}

Le informazioni posso essere trasmesse via cavo variando alcue proprietà fisiche (tensione/corrente). Rappresentando la tensione/corrente in una funzione f(t) è possibile analizzare il segnale. Fourier dimostrò che un segnale di questo tipo, periodico e abbastanza regolare può essere descritto da una ideale somma infinita di seni e coseni (\textbf{Serie di Fourier}). Una funzione può essere ricostruita  a partire dalla sua serie di Fourier. Analizziamo un segnale di trasmissione analogica e proviamo a ricostruirlo con Fourier. Dobbiamo tener conto che nessun canale trasmissivo è perfetto, per cui c'è sicuramente attenuazione. L'intervallo di frequenze trasmesse senza forte attenuazione è detto \textbf{banda passante}. Anche in un canale perfetto ci si accorge comunque che un segnale digitale non può essere trasmesso a velocità troppo elevate, esistono però alcuni schemi di codifica per aumentare la velocità di trasmissione.\\
Nyquist/Shannon dimostrò che la velocità massima di trasmissione è:\\

\[V_{max} = 2H\log_{2}V bit/sec\]

Mentre il livello di rumore si misura facendo il \textbf{rapporto segnale-rumore}. Solitamente viene indicata tale misura antecedendo \(10log_{10}\) e misurando in dB.

\[Segnale/Rumore = 10log_{10} S/N dB\]

Un risultato notevole ottenuto da Shannon fu:

\[MAX_{bit/s} = H log_{2}(1 + S/N)\]

con H pari all'ampiezza di banda in Hz.

\subsubsection{Mezzi Magnetici}

Sistema molto semplice, utilizzato da sempre e basato su un funzionamento banale: si salvano i dati su nastri magnetici (dischi rimovibili) e si trasportano fisicamente a destinazione dove verranno letti. Se si pensa a un tir che trasporta un centinaio di HD da 1TB che percorre qualche Km per consegnare questi dischi si può intuire che la larghezza di banda è elevatissima e con un costo irrisorio. La cosa banalmente poco buona è l'enorme ritardo nella trasmissione dati.

\subsubsection{Doppino}

Il doppino è composto da 2 conduttori di rame isolati attorcigliati tra loro a forma elicoidale (stile DNA), questo per evitare interferenze tra di loro (risulterebbero un ottima antenna). Viene largamente utilizzato nel sistema telefonico, questo perché il doppino può attraversare diversi Km senza bisogno di amplificare il segnale così dall'abitazione si può agevolmente arrivare alla centrale. Si possono usare per trasmettere dati Analogici o anche Digitali e la larghezza di banda dipende dal diametro del cavo e dalla distanza percorsa. Esistono più categorie di questi cavi che differiscono sostanzialmente per il numero di spire per cm per ridurre le interferenze. Il doppino cat3 (usato sino al 1988) sono composte da 2 cavi isolati cavi attorcigliati. Il doppino cat5 sono come i cat3 ma utilizzano più spire per cm questo li rendono più adatti a trasmissione ad alta velocità. Il doppino può arrivare a una banda di 250-600MHz. Questi cavi sono detti anche UTP (\textit{Unshielded Twisted Pair}).

\subsubsection{Cavo coassiale}

Essendo più schermato del precedente il cavo coassiale può estendersi per distanze maggiori. Esistono 2 tipi di cavi coassiali: da 50 Ohm per le trasmissioni digitali e da 75 Ohm per le analogiche.
Composto da un nucleo di rame, rivestito da materiale isolante a sua volta rivestito da una calza conduttrice il tutto ricoperto da una guaina protettiva, il cavo coassiale è caratterizzato da un eccellente immunità al rumore. L'ampiezza di banda di questi cavi arriva attorno a 1GHz e dipende dalla lunghezza, dalla qualità e dal rapporto segnale-rumore del segnale.

\subsubsection{Fibra Ottica}

Un sistema di trasmissione ottico è formato principalmente di 3 parti: sorgente luminosa, mezzo di trasmissione e rilevatore di luce. La sorgente di luce è rappresentata o da LED o semiconduttori laser. Il mezzo trasmissivo ovviamente è la fibra composta da un nucleo (core) di vetro di pochi micron avvolto in una guaina di vetro (cladding) con indice di rifrazione più basso e infine la solita rivestitura con guaina in plastica. La fibra si basa su un principio molto semplice, ovvero che la luce che la attraversa viene riflessa al suo interno fino ad arrivare all'altra estremità del cavo, questo avviene perchè la luce immessa nel core incontra il cladding con indice di rifrazione minore e il raggio luminoso viene così riflesso (se possiede un inclinazione corretta). La velocità di tale raggio è circa quella della luce infatti il limite di banda della fibra non è dovuto alla velocità di trasmissione ma di decodifica del segnale luminoso in impulso elettrico. Una fibra può contenere più raggi che si riflettono in essa l'importante è che il loro angolo di riflessione sia diverso. Questo tipo di fibre è detto \textbf{multimodale}. Le fibre che invece permettono la trasmissione di luce in linea retta sono le \textbf{monomodali} che non sono altro che guide d'onda ma possono raggiungere i 50Gbps per 100Km senza attenuazione.\\ Un problema delle fibre è il collegamento tra 2 i esse che può avvenire in 3 modi:

\begin{enumerate}

\item{ Le fibre vengono inserite in apposite prese grazie a dei connettori con perdita di circa 10-20\% della luce};
\item{ Le fibre vengono attaccate meccanicamente, messe di fronte una all'altra e poi viene avvolto in una macchina particolare per poi essere pinzate, con perdita comunque di circa il 10\% della luce};
\item{ Le fibre vengono fuse tra loro. Una soluzione quasi ottimale, anche se difficile, genera una piccola attenuazione di segnale}.

\end{enumerate}

Le LAN basate sulle fibre ottiche solitamente sono ad anello con congiunzione a T per ogni pc o a stella passiva. La configurazione ad anello ha il difetto che se una congiunzione si guasta salta tutta la rete.

\subsubsection*{Tabella riassuntiva}

\begin{figure}[htpb]
\centering
\includegraphics[scale=1]{images/mezzi.png}
\caption{Riassunto caratteristiche dei mezzi trasmissivi.}
\end{figure}

\newpage

\subsection{Trasmissioni Wireless}

\subsubsection{Lo spettro elettromagnetico}

Lo spostamento di elettroni crea campi magnetici. Questa osservazione fu fatta per la prima volta da Hertz. Il numero di oscillazioni al secondo di un'onda è chiamato \textbf{frequenza} e si misura in Hz. La distanza massima tra 2 picchi (o minimi) è chiamata \textbf{lunghezza d'onda}.
Tutte le trasmissioni wireless si basano sul principio che un antenna collegata a un circuito elettrico invia onde elettromagnetiche che possono essere captate da un ricevitore posto a una distanza appropriata. Le onde viaggiano nell'etere alla velocità della luce.
Qui di seguito è visualizzato lo spettro elettromagnetico e la sua suddivisione.

\begin{figure}[htpb]
\centering
\includegraphics[scale=1]{images/spettro.png}
\caption{Spettro elettromagnetico.}
\end{figure}

\subsubsection{Trasmissioni radio}

Le onde radio sono onde omnidirezionali semplici da riprodurre e viaggiano per lunghe distanze attraversando gli edifici. Non necessita di alcun allineamento trasmettitore-ricevente. Nell'aria l'attenuazione delle onde con frequenze più basse è di circa 1/\(r^2\). Nelle bande VLF, LF, MF le onde seguono la forma del terreno e possono viaggiare per circa 1000 Km. Nelle alte frequenze invece le onde che riescono ad entrare nella ionosfera vengono riflesse e ritornano sulla terra permettendo così di percorrere distanze notevoli. 

\subsubsection{Trasmissione a microonde}

Sopra i 100 MHz le onde viaggiano quasi in linea retta per cui è necessario un allineamento trasmettitore-ricevente. Concentrando l'onda in un piccolo raggio si ottiene un ottimo rapporto segnale/rumore. Il problema di queste trasmissioni sta nelle lunghe distanze e dalla curvatura della terra, la quale porta alla necessità di ripetitori. L'utilizzo di antenne molto alte riduce l'effetto della curvatura. Alcune onde posso rinfrangersi negli strati più bassi dell'atmosfera e arrivare in ritardo rispetto a quelle dirette e addirittura fuori fase causandone l'annullamento (\textbf{multipath fading}). Le microonde non attraversano gli edifici molto bene, e sono soggette alle condizioni climatiche. Comunemente si usano bande sopra i 10 GHZ ma sopra i 40 GHz la pioggia comincia ad assorbire le onde. Acquisto e installazione di apparecchi per questo tipo di trasmissione è molto basso.

\subsubsection*{Divisione dello spettro elettromagnetico}

Tutti vogliono un pezzo di spettro per aumentare la velocità di trasmissione e quindi bisogna regolarne tale divisione, se ne occupa l'ITU-T. In passato per compiere tale divisione sono stati utilizzati 3 modi:

\begin{enumerate}

\item{Concorso di bellezza: ognuno di coloro che voleva spettro doveva dare un motivo per il quale doveva averlo proprio lui. Problemi: corruzioni e scelte arbitrarie senza senso.}
\item{Lotteria: veniva fatta una vera e propria lotteria per assegnare lo spettro. Problemi: partecipavano anche i non interessanti solo per guadagnare soldi dalla rivendita dello spettro. }
\item{Asta: vendita all'asta dello spettro. Problema: banca rotta delle aziende. }
\item{Libertà: approcio che non prevede assegnamento di spettro, lasciando trasmissione libera ma regolata. Ovvero la potenza doveva essere utilizzata in modo da limitare la portata per evitare interferenze. }

\end{enumerate}

\subsubsection{Infrarossi}

Sistema economico e facile da costruire con il difetto di non riuscire ad attraversare gli ostacoli. Queste onde, infatti, si avvicinano alle onde di tipo luminose. Sono più sicuri delle onde radio per la difficoltà di intercettazione, ma risentono molto degli ostacoli. Questo a volte può rappresentare anche un vantaggio (es. telecomandi tv). Per questi motivi sono usati per distanze brevi (collegamento pc-stampanti, telecomandi ecc..), e hanno un ruolo secondario nelle telecomunicazioni. Il sistema infrarosso non richiede alcuna licenza governativa.

\subsubsection{Trasmissioni a onde luminose (LASER)}

Sistema poco costoso che offre una banda molto elevata, facile da installare e non richiede licenze. La sua debolezza sta nel raggio molto sottile (e unidirezioneale) e quindi difficile da indirizzare verso il bersaglio esatto. Per ovviare al problema a volte vengono inserite lenti per rendere il raggio meno focalizzato. Il raggio laser per di più non attraversa la pioggia e la nebbia ed è soggetto a fenomeni di convezione (turbolenza provocata da fonti di calore).

\newpage

\subsection{Satelliti}

Un satellite può essere immaginato come un grande ripetitore di microonde collocato nel cielo, che contiene molti \textbf{trasponder} (ricetrasmettitori satellitari). I raggi verso la terra possono essere più o meno grandi, questa modalità è detta \textbf{bent pipe}. I satelliti hanno un periodo orbitale che dipende dalla loro altezza rispetto alla terra. Un problema è la presenza delle \textbf{fasce di Val Allen}, strati di particelle molto cariche, che distruggerebbero un satellite in poco tempo.

\begin{figure}[htpb]
\centering
\includegraphics[scale=1]{images/Orbits.png}
\caption{Posizione dei salettiti GEO, MEO e LEO}
\end{figure}

\subsubsection{Satelliti Geostazionari (GEO)}	

Questi satelliti sono posti in orbite molto alte e con le tecnologie odierne non si possono collocare 2 satelliti GEO a meno di 2 gradi nel piano equatoriale, quindi con 180 satelliti si copre tutto. L'allocazione degli slot spaziali è gestito dall'ITU. L'ITU inoltre ha assegnato alcune bande di frequenza alle applicazioni satellitari in modo da non interferire con i sistemi a microonde preesistenti. I segnali inviata da questi satelliti viaggia alla velocità della luce, ma essendo molto lontani dalla terra hanno comunque un ritardo di circa 300 ms. I primi satelliti GEO con una singola emissione coprivano circa 1/3 della terra, chiamata \textbf{impronta}. Poi con lo sviluppo delle tecnologie si è cominciato a concentrare i raggi trasmissivi (spot) in aree geografiche più piccole (centinaia di Km). Un nuovo passo avanti nel settore delle comunicazioni satellitari si ebbe con le stazioni VSAT, piccole stazioni con una'antenna da circa 1m che comunicano con i satelliti GEO e per le loro piccole dimensioni e potenza non essendo in grado di comunicare tra loro si è dovuto ideare  alcune stazioni particolari più potenti per fare da ponte. Sono ovviamente mezzi di trasmissione broadcast e per quanto riguarda la sicurezza sono un disastro. Il costo della trasmissione satellitare non dipende dalla distanza, ma è costante. Hanno una reattività quasi istantanea e un ottimo tasso di errore.

\subsubsection{Satelliti su orbite medie (MEO)}

Questi satelliti si muovono sopra di noi a una velocità relativamente bassa percorrendo il giro del pianeta in circa 6 ore. Coprono un area più piccola dei GEO e si possono raggiungere con mezzi meno potenti. I 24 satelliti GPS che orbitano a 18000 Km sono di tipo MEO.

\subsubsection{Satelliti su orbite basse (LEO)}

Si spostano molto velocemente e per realizzare un sistema completo sono necessari molti satelliti di questo tipo. Essendo vicini alla crosta terrestre le stazioni non hanno bisogno di molta energia per comunicare con poco ritardo.

\subsubsection*{Iridium}					

Lanciati nel 1997 i 66 satelliti LEO del progetto Iridium (Motorola) vennero acquistati da un investitore riprendendo il servizio nel marzo 2001, che era stato fermato nel 1999. Il progetto fornisce un servizio di telecomunicazione a livello mondiale basato su ``cellulari'' particolari'. I satelliti Iridium sono collocati a 750 Km di altezza, con un satellite ogni 32 gradi. Le trasmissioni avvengono nello spazio: ogni satellite comunica con altri satelliti fino a destinazione.

\subsubsection*{GlobalStar}

Basato su 48 satelliti LEO utilizzando uno schema diverso dal precedente. Il satellite che riceve la chiamata trasmette a una centrale terrestre che comunica con altre fino al satellite posto sulla cella del destinatario. La complessità resta quindi terrestre facilitandone la gestione.

\subsubsection*{Teledesic}

Progetto mirato per gli utenti internet, con l'idea di offrire 100 Mbps in trasmissione e 720 in ricezione. Il sistema è composto da 32 satelliti LEO con un impronta più grande. Basato sulla commutazione di pacchetto, con ogni satellite in grado di instradare ogni singolo pacchetto.

\subsubsection{Satelliti o fibra?}

E' ovvio che la comunicazione con le fibre sia molto veloce ma ci sono molti settori in cui i satelliti non possono essere rimpiazzati dalla fibra, come ad esempio la comunicazione mobile. Le comunicazioni broadcast per eccellenza sono satellitari. Un altro settore è la comunicazione in luoghi ostili, in cui le infrastrutture terrestri scarseggiano e la posa di cavi ottici non sarebbe il massimo. Inoltre anche le zone in cui i costi di posa sono elevati il satellite può essere un ottima alternativa. In sostanza per la comunicazione terrestre è sicuramente migliore la fibra ma il satellite rimarrà indispensabili per altri settori.

\subsection{Sistema Telefonico}

\subsubsection{Struttura della rete}

Inizialmente il mercato dei telefoni prevedeva la vendita di 2 apparecchi e spettava all'utente tirare il cavo tra i 2 telefoni. Questo intuitivamente creò una struttura di rete troppo confusa. Bell notato questo particolare aprì il primo ufficio di commutazione nel 1878. Le chiamate quindi dovevano passare per la centrale nella quale un addetto si occupava di collegare con un cavo il chiamante al chiamato. La rete però, sebbene meno complessa, rimase ancora troppo confusa poichè non era pensabile collegare ogni ufficio di commutazione a una centrale. Vengono così ideati i livelli delle centrali di commutazione. Inizialmente con 2 livelli fino ad arrivare a 5. In generale una comunicazione avviene a più livelli:

\begin{enumerate}

\item{La richiesta di chiamata arriva alla centrale locale del chiamante alla quale è collegato direttamente con 2 cavi di rame (doppino di categoria 3). Se il chiamato appartiene alla stessa centrale locale avviene il collegamento tra le parti};
\item{La centrale locale è collegata a una centrale interurbana (con cavi in fibra/microonde/coassiale),e come per quella locale se la centrale locale del chiamato è collegata alla stessa centrale interurbana allora le parti si collegano};
\item{Le centrali interurbane sono connesse a centrali intermedie e la trasmissione avviene analogamente alle precedenti}.

\end{enumerate}

Le trasmissioni sono preferibili in digitale per la non necessità di accuratezza, per il basso costo e per la semplicità di gestione.
Si possono quindi individuare 3 componenti fondamentali del sistema telefonico:

\begin{enumerate}

\item{Collegamenti locali: rappresentano il collo di bottiglia del sistema};
\item{Linee: collegamenti in fibra tra le centrali};
\item{Centrali di commutazione: che spostano le chiamate tra le linee}.

\end{enumerate}

\subsubsection{Modem, ADSL, Wireless}

Il modem ha la funzione di convertire i dati dalla forma digitale del pc alla forma analogica necessaria per inviare i dati attraverso un collegamento locale. Nella centrale poi i dati vengono ritrasformati in digitale e poi ritrasmessi in linee a lunga distanza. Dall'altro capo poi ci sarà il modem per la conversione inversa alla precedente. I problemi principali delle linee di trasmissioni sono 3:

\begin{enumerate}

\item{Attenuazione: rappresenta la perdita di energia (in dB/Km) dalla propagazione del segnale. Tale perdita dipende dalla frequenza};
\item{Distorsione: rappresenta la differenza di velocità tra le varie componenti del segnale};
\item{Rumore: rappresenta energia indesiderata all'interno del segnale originale, causate da sorgenti di trasmissione esterne}.

\end{enumerate}

\begin{figure}[htbp]
\centering
\includegraphics[scale=1]{images/phone.png}
\caption{Percorso tipico di una chiamata a media distanza}
\end{figure}

\subsubsection*{Modem}

Per riuscire a inviare dati in forma digitale è necessario un ampio spettro di frequenza questo rende adatta la trasmissione in banda base (DC) solo a basse velocità e distanze brevi. Il problema è aggirato utilizzando una trasmissione (AC) aggiungendo un segnale portante tra i 1000 e i 2000 Hz. Nella modulazione di ampiezza (ASK) sono utilizzate 2 diverse ampiezze 0 e 1, nella modulazione di frequenza (FSK) si utilizzano 2 o più toni. In quella di fase invece l'onda portante è spostata di 0 o 180 gradi a intervalli regolari.

\begin{figure}[htbp]
\centering
\includegraphics[scale=1]{images/modulazione_ask.jpg}
\caption{Modulazione d'ampiezza}
\end{figure}

\begin{figure}[htbp]
\centering
\includegraphics[scale=1]{images/modulazione_fsk.jpg}
\caption{Modulazione di frequenza}
\end{figure}

\begin{figure}[htbp]
\centering
\includegraphics[scale=1]{images/modulazione_psk.jpg}
\caption{Modulazione di fase}
\end{figure}

Un apparecchio che utilizza uno di questi metodi per ``tradurre'' un flusso di bit in segnale analogico è detto modem. La maggior parte dei modem campiona 2400 volte al secondo. Il numero di campionamenti al secondo si misura in \textbf{baud}. Durante ogni baud è trasmesso un simbolo.
Concetti da ricordare:

\begin{itemize}
\item{Banda passante: intervallo di frequenza passante nel mezzo con un attenuazione minima(Hz)};
\item{Baud rate: numero di campioni per secondo (=frequenza simboli)};
\item{Modulazione: determina il numero di bit per simbolo};
\item{Frequenza di bit: quantità di (simboli/sec)*(bit/simbolo)}.

\end{itemize}

La trasmissione digitale è adatta in banda base (DC) a basse velocità. Per aggirare i problemi di tale banda si usa la trasmissione AC introducendo un segnale costante detto \textbf{portante d'onda sinusoidale}. La sua ampiezza, frequenza o fase posso essere modulate per inviare informazioni. Nella \textbf{modulazioni di ampiezza} vengono usate 2 ampiezze diverse per rappresentare 1 e 0 mentre in quella di \textbf{frequenza} (FSK) si utilizzano più toni. Infine nella \textbf{modulazione di fase} più semplice l'onda viene spostata di 0 o 180 gradi (schemi migliori utilizzano spostamenti più piccoli). I modem odierni utilizzano modulazioni ibride per avere un maggior baud rate. Un esempio è la QPSK (\textit{Quadrate Phase Shift Keying}). Questa tecnica di modulazione prevede l'utilizzo di più fasi e più ampiezze, e in base al numero di combinazioni nascono i nomi QAM-16 (4 bit per simbolo utilizzando 4 fasi e 4 ampiezze, 9600 bps), QAM-64 (\textit{Quadrature Amplitude Modulation})e così via.

\begin{figure}[htbp]
\centering
\includegraphics[scale=1]{images/qam.png}
\caption{Diagramma costellazione QAM}
\end{figure}

Uno schema con costellazione molto fitta è soggetto a errori per questo i modem che li adottano utilizzano meccanismi di correzione degli errori, per esempio con un bit extra di parità.
Gli standard utilizzati dai modem più conosciuti sono:

\begin{itemize}

\item{V.32: trasmette 4 bit più 1 di parità a 2400 baud (9600 bps)};
\item{V.32 bis: trasmette 6 bit più 1 di parità a 2400 baud (14400 bps)};
\item{V.34: utilizza 12 bit per simbolo a 2400 baud (28800 bps)};
\item{V.34 bis: utilizza 14 bit per simbolo a 2400 baud (33600 bps)}.

\end{itemize}

Una connessione che permette ai dati di viaggiare in entrambe i sensi è detta \textbf{full duplex}, mentre se lo permette ma solo uno alla volta è detta \textbf{half duplex} se invece è permesso un solo senso è \textbf{simplex}. In base quanto detto la velocità massima dei modem è 34 Kbps questo non è vero perché l'ampiezza del canale telefonico è 4 MHz quindi per Nyquist il numero di campioni massimo è 8000, per gli 8 bit per campione usato negli U.S. si hanno 64 Kbps! In realtà non è così perché 1 bit serve per il controllo per cui si riduce il tutto a 56 Kbps che è lo standrard\textbf{ V.90}.

\subsubsection*{Linee DSL}

Servizi con banda maggiore a quella appena descritta sono detti a \textbf{banda larga}. Per ottenere questo aumento di banda viene usato un artificio che si basa sostanzialmente sulla rimozione del filtro che limitava la capacità del collegamento locale a 3100 Hz. Questi servizi, detti xDSL, sarebbero dovuti funzionare sui doppini già installati nelle abitazioni, senza creare problemi ai telefoni, con costi limitati e non legati al tempo di utilizzo e ovviamente dovevano velocizzare di molto quei 56 Kbps. 

\begin{itemize}

\item{Proposta AT\&T: divisione dello spettro delle reti locali in POTS, upstream e downstream};
\item{DMT (\textit{Discrete MultiTono}): lo spettro è diviso in 256 canali indipendenti. Il primo canale utilizzato per POTS, i 5 successivi non vengono utilizzati per limitare le interferenze e tutti gli altri per i dati, e chi fornisce il servizio decide come dividerli tra up e down (solitamente 10\%-90\%)}.

\end{itemize}

Quest'ultima idea fa nascere l'ADSL che può arrivare a circa 8 Mbps in ricezione e 1 Mbps in trasmissione. In realtà questi valori sarebbero maggiori ma il rapporto segnale/rumore non li permettono. Per usufruire dell'ADSL è necessario installare un \textbf{NID} e uno \textbf{splitter} spesso all'interno di uno stesso pezzo, questo per filtrare le bande e un modem. Se quest'ultimo non è interno al pc allora si deve collegare a esso con ethernet, USB o rete wireless.

\subsubsection{Multiplexing}

Per limitare i costi le aziende hanno ideato modi per convogliare più conversazioni nello stesso mezzo fisico, appunto il\textbf{ Multiplexing}. Esistono sostanzialmente 2 categorie di quest'ultimi:

\begin{itemize}

\item{FDM: Frequency Division Multiplexing};
\item{TDM: Time Division Multiplexing};

\end{itemize}

In FDM lo spettro fi frequenze è diviso in bande e ogni utente ne possiede una. In TDM gli utenti si danno il cambio, tipo round-robin.

\subsubsection*{Multiplexing a divisione di frequenza}

La banda è limitata dai filtri a 3.1 KHz, e nell'unione in multiplexing viene allocato uno spazio un po superiore, 4 KHz per avere un po di tolleranza. Poi ogni canale voce viene aumentato di una frequenza diversa e unito agli altri senza sovrapposizione. Nonostante questi accorgimenti 2 canali adiacenti avranno u po di sovrapposizione che potrebbe tramutarsi in un po di rumore nei 2 canali. Uno standard comune prevede 12 canali voce uniti in multiplexing nella banda tra 60-108 KHz. Questa unità è chiamata \textbf{gruppo} che possono essere uniti in multiplexing in un \textbf{supergruppo} e a loro volta ancora in un \textbf{mastergroup}.

\subsubsection*{Multiplexing a divisione di lunghezza d'onda}

Utilizzato per i canali in fibra, il \textbf{WDM} (\textit{Wavelenght Division Multiplexing}) si fonda sul principio della combinazione e divisione di lunghezze d'onda. Più fibre vengono combinate convogliando ogni segnale in un unico canale nella cui estremità c'è uno splitter utile a ripristinare i segnali delle fibre di partenza. La differenza sostanziale rispetto all'FDM è il sistema ottico completamente passivo. Un sistema con molti canali e lunghezze d'onda ravvicinate è definito \textbf{DWDM} (Dense WDM).

\subsubsection*{Multiplexing a divisione di tempo}

Gestita completamente da dispositivi elettronici digitali, per cui è necessaria una conversione da parte della centrale prima di trasmettere il segnale sulla linea di uscita. La centrale locale digitalizza il segnale analogico producendo numero a 8 bit (grazie al \textbf{codec}, \textit{coder-decoder}), elaborano  8000 campioni al secondo. Questa tecnica è chiamata PCM e costituisce il cuore del sistema telefonico odierno. Nel mondo esistono molti schemi PCM diversi incompatibili tra loro. In Giappone e in Nord America si utilizza la portante T1 in altre zone del mondo quella E1. Una tecnica chiamata \textbf{differential pulse code modulation} al posto di inviare l'ampiezza digitalizzata invia la differenza rispetto alla precedente così da ridurre il numero di bit (da 7 a 5) utili supponendo che sia poco probabile il salto di \(\pm\)16. Una variante di questa tecnica detta \textbf{modulazione delta} si basa su un principio simile: ogni valore campionato differisce dal precedente di \(\pm\)1 sotto le condizioni che può essere trasmesso un singolo bit che dice se il nuovo campione è maggiore o minore del precedente. Questa tecnica che ipotizza una bassa variazione di segnale può avere problemi con bruschi cambiamenti di livello. Esistono altre tecniche dette \textbf{codifiche per ipotesi} che utilizzando pochi valori precedenti prevedono il successivo.

\subsubsection{Commutazione}

\subsubsection*{Commutazione di circuito}

Quando viene avviata una telefonata l'apparecchio di commutazione del sistema telefonico prova a creare un percorso fisico tra il chiamante e il chiamato. 

\subsubsection*{Commutazione di messaggio}

Questa tecnica non prevede un collegamento fisico a priori ma si basa su un idea diversa, ovvero un passo alla volta. Il messaggio viene inviata alla prima centrale di commutazione, la quale dopo averlo esaminato per vedere gli eventuali errori, lo ritrasmette alla successiva fino ad arrivare al destinatario. Questa tecnica è chiamta di \textbf{store and forward}.

\subsubsection*{Commutazione di pacchetto}

Questa tecnica è molto diversa dalle precedenti e si basa sull'idea di dividere i dati in pacchetti limitati i quali partono e possono arrivare anche in ordine sparso sarà compito del destinatario riordinarli. Non c'è bisogno di alcun collegamento predefinito e ogni pacchetto può percorrere strade diverse. E' più resistente agli errori della commutazioni di circuito , poiché si possono aggirare commutatori bloccati passando per un altro percorso. Inoltre la commutazione di pacchetto non riserva alcuna ampiezza di banda per cui in linea generale è più efficiente. L'addebito dipende sia dal tempo che dalla distanza.

\begin{figure}[htbp]
\centering
\includegraphics[scale=0.8]{images/comm.png}
\caption{Confronto tra commutazioni}
\end{figure}

\subsection{Sistema telefonico mobile}

Esistono 3 generazioni di telefoni cellulari:

\begin{enumerate}

\item{Voce analogica};
\item{Voce digitale};
\item{Voce e dati digitali}.

\end{enumerate}

\subsubsection{Cellulari di I generazione}

Il primo esempio di ``cellulare'' lo si ha nel 1946 quando venne creato il sistema premi e parla, come ad esempio quella dei CB. Negli anni sessanta scompare il tasto per parlare grazie all' IMTS (\textit{Improved Mobile Telephone System}) che utilizzava un trasmettitore ad alta potenza posto in una collina, il quale utilizzava 2 frequenze, una per la ricezione e una per trasmettere. IMTS utilizzava solo 23 canali distribuiti tra 150 e 450 MHz. Il numero limitato di canali faceva si che alcuni utenti dovevano aspettare molto prima di aver segnale libero.

\subsubsection*{Sistema telefonico mobile avanzato}

Cambiò tutto grazie a AMPS (\textit{Advanced MPS}). Ogni area geografica era divisa in \textbf{celle}, in AMPS grandi 10-20 Km. Ogni cella utilizzava un insieme di frequenze diversa da quelle vicine. L'utilizzo di celle piccole richiede meno potenza. L'idea principale sta proprio qui, in celle piccole e grande riutilizzo delle frequenze. Nelle aree in cui il numero di utenti è elevato e il sistema tende a sovraccaricarsi ,le celle vengono a loro volta divise in \textbf{microcelle} così da aumentare il riuso delle frequenze. Tanto più piccole sono le celle tanto meno potenti devono essere i dispositivi. Da notare il fatto che una frequenza utilizzata da una cella non è più usata nell'area cuscinetto attorno ad essa (area di circa 2 celle). Al centro di ogni cella si trova una stazione la quale è collegata a un dispositivo chiamato MTSO (\textit{Mobile Telephone Switching Office}) o MSC(Mobile Switching Center). In sistemi più grandi sono necessari più MTSO che quindi vengono divisi in livelli. Ogni MTSO colloquia con gli altri. Un telefonino in ogni istante è logicamente posizionato in una certa cella e ogni qualvolta il segnale in tale cella si affievolisce, la stazione base colloquia con le adiacenti per delegare la gestione dell'apparecchio alla cella col segnale più forte. Questo processo è chiamato \textbf{handoff} e richiede 30 msec. Esistono 2 tipi di handoff:

\begin{itemize}

\item{soft handoff: l'acquisizione della nuova stazione avviene prima di interrompere il segnale precedente};
\item{hard handoff: la vecchia stazione rilascia il telefono prima che la nuova lo acquisisca. La chiamata viene bruscamente interrota}.

\end{itemize}

\subsubsection*{Gestione della chiamata}

Ogni telefono AMPS ha un numero seriale di 32 bit e un numero di telefono 10 cifre. Ogni volta che viene acceso, il telefono esplora i vari canali e trova il segnale più potente. Il telefono quindi trasmette in broadcast il proprio seriale e il numero di telefono con un codice di correzione degli errori. La stazione base aggiorna l' MTSO e ogni 15 minuti circa aggiorna la posizione corrente. Per chiamare il telefono acceso invia il numero del chiamato e i propri dati attraverso il canale di accesso e quando riceve la richiesta la stazione base informa l'MTSO. Se il chiamante appartiene a quell'MTSO cerca un canale libero per la chiamata, e trasmette il numero del canale al telefono. Il processo di ricezione è diverso: ogni telefono è in ascolto nel canale di trasferimento e quanto l'MTSO riceve il pacchetto che richiede il destinatario lo passa alla stazione base la quale chiede conferma al telefono. In caso affermativo la stazione invia il numero del canale con la chiamata e inizia la conversazione.

\subsubsection{Cellulari di II generazione}

Nel mondo sono sostanzialmente utilizzati  4 sistemi: D-AMPS, GSM, CDMA e PDC utilizzato solo in Giappone e molto simile al D-AMPS.

\subsubsection*{D-AMPS (\textit{Digital} AMPS)}

Il D-AMPS è totalmente digitale. Progettato per coesistere con AMPS utilizza gli stessi canali a 30 KHz con le stesse frequenze. Si è resa disponibile una nuova banda di frequenza 1850-1910 MHz per sostenere l'aumento del carico. Alcuni cellulari erano in gradodi utilizzare entrambe le bande disponibili. Su un telefono D-AMPS il segnale voce preso dal microfono viene digitalizzato e compresso dal \textbf{vocoder}, questa compressione permette la condivisione di una coppia di frequenze (upstream/downstream) fino a 3 utenti con multiplexing a divisione di tempo. Ogni coppia di frequenza supporta 25 frame/sec di 40 msec. Ogni frame è diviso in 6 slot temporali. Gruppi di 16 frame costituiscono un superframe, con alcune informazioni di controllo. Concettualmente funziona come AMPS: viene acceso il telefono, viene contattata la stazione e poi rimane in ascolto.
Nei tempi in cui il cellulare non riceve ne trasmette viene testata la qualità della linea. Questa tecnica è chiamata \textbf{MAHO}(\textit{Mobile Assisted HandOff})

\subsubsection*{Comunicazioni GSM \textit{(Global System for Mobile communications)}}

Molto simile ad D-AMPS però ha i canali più ampi così possono supportare ben 8 utenti in una coppia di frequenze. Un sistema GSM ha 124 coppie di canali simplex ampi 200 KHz e supporta 8 connessioni grazie a multiplexing a divisione di tempo. A ogni stazione attiva è assegnato uno slot temporale su una coppia di frequenze. Quindi teoricamente supporta 992 canali molti di questi però utilizzati come canali di controllo. Trasmissione e ricezione non avvengono nello stesso intervallo di tempo perché il sistema non è in grado di gestirlo. Questo protocollo ha introdotto le schede SIM che contengono al loro interno IMSI (identifica la SIM) e la chiave di crittografi (Ki). L'identificazione avviene così:

\begin{enumerate}

\item{Il cellulare manda Ki e IMSI in broadcast};
\item{L'operatore lo riceve e manda un numero casuale};
\item{Il cell lo rimanda firmato con Ki};
\item{L'operatore controlla}.

\end{enumerate}

Il \textbf{canale di controllo broadcast} è un flusso continuo di dati trasmessi dalla stazione base che annuncia identità e stato del canale. Il \textbf{canale di controllo dedicato} è utilizzato per aggiornare la posizione, registrare il terminale nella rete e configurare la chiamata. Infine c'è un \textbf{canale di controllo comune} che è diviso in 3 sottocanali logici. Il primo è il \textbf{canale di paging} utilizzato dalla stazione per annunciare le chiamate in arrivo. Poi c'è il \textbf{canale ad accesso casuale} e permette agli utenti d richiedere uno slot sul canale di controllo dedicato. Infine c'è il \textbf{canale di assegnazione dell'accesso} che assegna il lo slot del canale di controllo dedicato per ripetere le richieste efettuate dal secondo canale.

\subsubsection*{CDMA \textit{(Code Division Multiple Access)}}

Miglior sistema rispetto a quelli presentati e base per la III generazione a volte chiamato \textbf{cdmaOne}. CDMA permette la trasmissione per tutto il tempo attraverso l'intero spettro. Queste trasmissioni multiple simultanee vengono separate tramite tecnica di codifica. L'idea sta nel fatto che i segnali si sommino linearmente. Per cui tutti comunicano ma ogni coppia lo fa in ``lingua diversa''. Per risalire a ciò che viene detto basta togliere il rumore aggiunto dalle conversazioni di altri. Per riuscire a filtrare tale segnale rumoroso vengono utilizzate le matrice di Hadamard. Tecnicamente il CDMA funziona così:\\ ogni tempo di bit è diviso in m intervalli chiamati \textbf{chip} (generalmente 64-128 chip per bit) e ad ogni stazione viene assegnata una \textbf{sequenza di chip} univoca. Per trasmettere un 1 la stazione deve semplicemente inviare tale sequenza se invece invia uno zero deve farne il complemento. Non sono ammessi altri schemi. Per aumentare la quantità di informazione inviabile basta passare da b bit/sec a mb chip/sec aumentando l'ampiezza di banda di un fattore m. CDMA è una forma di comunicazione a spettro distribuito. Ognuna di queste sequenze di chip sono mutualmente ortogonali, ovvero ogni prodotto interno normalizzato di qualunque coppia di sequenze è uguale a 0 (ottenute con i \textbf{codici Walsh}). 
\[ S*T = \frac{1}{m}\sum\limits_{i=1}^m S_i T_i = 0\]
Da cui si deduce che S*S = 1 e S*\(\overline{S}\) = -1. 
Ora, ogni stazione invia queste sequenze come bit le quali si ``mischiano'' con le altre, tecnicamente si sommano con gli altri segnali di altre stazioni. Una volta che il segnale arriva alla stazione di destinazione per sapere il bit inviato dalla sorgente basterà moltiplicare il segnale per la sequenza di chip della sorgente e si otterrà il bit inviato.
Matematicamente se si deve capire il messaggio C allora sarebbe \textbf{S = A + \(\overline{B}\) + C}:
\[S*C = (A+ \bar{B} + C)*C = A*C + \bar{B}*C + C*C = 0 + 0 + 1 = 1\]
Per la proprietà di ortogonalità tutti i prodotti si sono annullati a parte quello interessato. Questo sistema è in genere utilizzato per reti wireless.

\subsubsection{Cellulari di III generazione}

La prima proposta di cellulari di III generazione fu fatta dall'ITU con l'intenzione di lanciarli nell'anno 2000 con ampiezza di banda di 2 MHz e 2Mbps per tutti. Un sogno irrealizzabile che ha visto il tutto slittare qualche anno più avanti e con alcune specifiche smussate come i 2Mbps per chi stava fermo e circa 400 per gli utenti che camminavano e 144 per quelli che si spostano a più alte velocità. I servizi che avrebe dovuto fornire IMT-2000 (così si chiamva la proposta) erano:

\begin{itemize}

\item trasmissione voce ad alta qualità;
\item trasmissione messaggi;
\item applicazioni multimediali;
\item accesso internet.

\end{itemize}

Per riuscire ad utilizzare questi servizi in tutto il mondo si era pensati di creare un unica tecnologia per rendere tutto più semplice. Furono fatte diverse proposte.

\subsubsection*{W-CDMA (\textit{Wideband CDMA})}

Questo protocollo fu proposto da Ericsson. In Europa battezzato col nome \textbf{UMTS} \textit{(Universal Mobile Telecommunications System)}. Si basa sui fondamenti del CDMA utlizzando però una banda larga a 5 MHz ed è stato progettato per interagire con il sistema GSM anche se non compatibile. Data rate 384 Kbps.

\subsubsection*{CDMA2000}

Proposto da Qualcomm, simile al precedente con la differenza di non interagire con GSM. Altre differenze col precedente sono il tempo di frame, di spettro e una diversa tecnica di sincronizzazione. Data rate 144 Kbps.

\subsubsection*{EDGE \textit{(Enhanced Data for GMS Evolution)}}

Uguale al GSM con un numero maggiore di bit per baud che comportano però più errori per baud. Sistema pensato durante il passaggio da II a III generazione, definito infatti 2.5G.

\subsubsection*{GPRS \textit{General Packet Radio Service}}

Altro schema per 2.5G. Una rete di pacchetti costruita sopra ad D-AMPS e GSM. GPRS permette di inviare e ricevere pacchetti IP in una cella basata su sistema vocale. Quando attivo alcuni slot temporali vengono dedicati al traffico dei pacchetti. Questi slot sono divisi in canali logici. Ogni canale è utilizzato per scaricare i pacchetti nei quali c'è indicato il destinatario. Per inviare un pacchetto la stazione mobile richiede uno o più slot e effettua la richiesta alla base. La base poi invia il pacchetto via internet tramite rete via cavo.

\subsubsection{Oltre al 3G}

\subsubsection*{HSDPA \textit{(High Speed Downlink Packet Access)}}

\subsubsection*{HSUPA \textit{(High Speed Uplink Packet Access)}}

\subsubsection*{HSOPA \textit{(High Speed OFDM Packet Access)}}

\newpage